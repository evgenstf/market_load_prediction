\section{Аннотация}
В этой работе рассматривается архитектура типичной биржи и предлагаются методы прогнозирования торговой нагрузки на нее путем анализа данных,
доступных как со стороны организатора,
так и со стороны участников торгов.

Мы коснемся основных определений финансовых терминов и торговых механизмов, необходимых для полного понимания ситуации, обсудим актуальность предложенной темы и реализуем саму биржу. После будут  построены математические модели, которые предсказывают повышение торговой нагрузки для увеличения вычислительных мощностей на стороне внутренней архитектуры биржевого движка.

Полученное в процессе исследования решение позволяет снизить риск прекращения торгов в периоды повышенной волатильности.\\

This paper examines the architecture of a typical exchange and offers methods for predicting the trading load on it by analyzing data,
available both from the organizer's side,
the same applies to bidders.

We will touch on the main definitions of financial terms and trading mechanisms necessary for a full understanding of the situation, discuss the relevance of the proposed topic and implement the exchange itself. After that, mathematical models will be built that predict an increase in the trading load to increase the computing power on the side of the internal architecture of the exchange engine.

The solution obtained during the research process reduces the risk of trading termination during periods of high volatility.
\\

Keywords — finance, trading, load prediction, market, exchange, trading infrastructure

\pagebreak
